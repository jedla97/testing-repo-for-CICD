\section{Hierarchie výpočetní složitosti. Třídy složitosti P, NP, \#P, PSPACE, EXP, NP-těžké. Definice problému Problém batohu (Knapsnack), problém směrování vozidel (VRP), Metrický k-střed.}

\subsection{Polynomická vs exponenciální časová složitost.}

Algoritmy s polynomickou složitostí lze efektivně řešit, bez vynaložení velkého výpočetního výkonu. 
S délkou vstupu \textit{n} roste lineárně potřebný čas ($n^x$) například hledání nejkratší cesty. 
Algoritmy s exponenciální složitostí lze efektivně řešit pouze pro malé exponenty, bez potřeby velkého výpočetního výkonu. 
S délkou vstupu \textit{n} roste exponenciálně potřebný čas ($x^n$) například problém obchodního cestujícího.

\subsection{Hierarchie výpočetní složitosti.}

Hierarchie výpočetní složitosti slouží k charakterizaci algoritmů u kterých lze pouze nepřímo určit asymptotickou složitost. 
Využívá se třídní hierarchie. 
Třídy od nejjednoduší po nejsložitější jdou v tomto pořadí: P $\subseteq$ NP $\subseteq$ PH $\subseteq$ \#P $\subseteq$ PSPACE $\subseteq$ EXP (NSPACE) $\subseteq$ NP-složité $\subseteq$ unsolvable. 
Jednoduší třída je vždy podmnožinou složitější (do složitější patří i ty jednoduší).

\subsection{Třída P}

Třída P obsahuje polynimiální problémy. 
Tyto problémy se dají vyřešit v polynomiálním čase na deterministickém Turingově stroji (TS). 
Příkladem může být nalezení nejkdatší cesty, minimální kostry v grafu.

\subsection{Třída NP}

Třída NP obsahuje nedetermisistické polynomiální problémy. 
Tyto problémy lze řešit v polynomiálním čase na nedetermistickém TS (doposud nesestaven).

\subsection{Třída NP-complete (NP-úplné)}

NP-úplné (NPC) je podskupina problémů třídy NP, která se zabývá rozhodovacími problémy.
Jejich řešení je nejtěžší a všechna známá řešení lze na determistickém TS provést pouze s exponenciálním čase. 
Pro polinomiální nebylo zatím nalezeno řešení, ale zároveň nebylo dokázáno, že řešení neexistuje.

\textbf{Problém NPC}\,--\,pokud by se podařilo, převést jeden NPC problém na polynomiální čas tak to znamená, že každý NPC problém lze převést do polynomiálního. 
To by vedlo k prokázání P $=$ NP.

Bylo dokázáno, že každý algoritmus pro NPC problém lze použít k řešení jiného NPC problému.
Přibližně existuje 10\,000 známých NPC problémů.
Příkladem může být\,--\,Barvení grafů, Knapsnack (problém batohu), 3-partition problém (rozdělti množinu čísel podmnožiny o velikosti 3, které mají stejný součet)

\subsection{Třída NP-těžké}

O problému řekneme, že je NP-težký, jestliže se na něj redukuje NPC problém, ale zároveň nevíme jestli spadá do NP.
Redukce znaméná že se dá převést/konvertovat.

\subsection{P vs NP}

P je podmnožina NP jelikož každý problém lze řešit v polynomiálním čase na
 nedetermistickém TS.
Existuje matematický problém P $=$ NP, který dosud nebyl potvrzen nebo vyvrácen. Takže se přesně nedá určit jestli je P $=$ NP. 
Obecně se považuje, že P $\neq$ NP.

\subsection{Třída \#P}

Je podmnožina NP

\subsection{Třída PSPACE}

\subsection{Třída EXP}

\subsection{Knapsnack}

Existuje osoba (zloděj), který má batoh/zavazadlo.
Zavazadlo má určitou maximální hmotnost/velikost.
Zloděj se snaží do batohu vložit co nejvíce zboží v co největší hodnotě. 
Každé toto zboží má jinou hmotnost/velikost a jinou hodnotu.
Takže zloděj hledá co nejoptimálnější zboží, které může vložit do batohu a tím mít co největší profit.

\subsection{Problém směrování vozidel (VRP)}

Existuje 1 místo (sklad), která má x vozidel.
Vozidla ze skladu musí objet všechny místa a dovést tam zboží. 
Vozidla nemusí navštivit stejný počet míst, ale množina vozidel musí navštívit všechna místa co nejoptimálněji. 
Při jednom vozidle podobné problému obchodního cestujícího (TSP), jen s rozdílem, že u TSP si vybíráme začátek, ale u VRP máme začátek přesně definován.


\clearpage
\section{Problém obchodního cestujícího a modifikace genetických algoritmů, Genetické programování, Optimalizace hejnem, Optimalizace mravenčí kolonií, Evoluční strategie.}

navštívit každý vrchol a minimalizovat cestu
patří do NP-težký ale při tranformaci do rozhodovacího problému spadá do NPC 
spadá do teorie grafů a cílem je nalést hamiltnovský cyklus


\clearpage
\section{Definice grafu. Incidenční matice, matice sousednosti. Handshaking lemma. Algoritmus detekce bipartitního grafu. Silně propojené komponenty. Kosarajův algoritmus. Tarjanův algoritmus.}


\clearpage
\section{Vlastnosti grafu: průměr, excentricita. Párování grafu. Maďarský algoritmus. Problém časové tabule. Algoritmus barvení grafu. Izomorfismus grafu a Ullmanův algoritmus.}


\clearpage
\section{Problém maximálního toku a minimálního řezu grafem. Řešení problému s více zdroji a více cíly. Ford Fulkersonův algoritmus. Definice úzkého hrdla. Definice reziduální cesty.}


\clearpage
\section{Univerzální aproximační funkce. Dopředná neuronová síť. Maticová reprezentace NN. Gradientní sestup. Vrstva zahazování. Aktivační funkce. Softmax.}


\clearpage
\section{Konvoluční neuronové sítě – princip. Max pooling, Dávková normalizace. Známé architektury neuronových sítí.}


\clearpage
\section{Lineární regrese. Polynomiální regrese. Logistická regrese.}


\clearpage
\section{Rekurentní neuronové sítě. LSTM. UNet sítě. Struktury neuronových sítí.}


\clearpage
\section{Q-učení, srovnání s genetickými algoritmy.}

\clearpage
\section{Extrakce znalostí ze stromových a grafových struktur. Metoda náhodného průchodu. Node2Vec. Obecná umělá inteligence – relační induktivní zaměření, kombinatorická generalizace. Předávání zpráv.}

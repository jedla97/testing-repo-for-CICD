\section{Formální definice kryptografického systému, symetrické a asymetrické šifry. Výpočetně těžké matematické problémy pro asymetrické šifry.}


\clearpage
\section{Služby bezpečnosti zajišťované kryptografickými prostředky, kryptografické mechanismy, které tyto služby zajišťují.}


\clearpage
\section{Kryptograficky bezpečné generátory náhodných čísel – požadavky, hodnocení bezpečnosti, příklady realizace.}


\clearpage
\section{Hašovací funkce - požadavky, hodnocení bezpečnosti. Princip konstrukce - iterační, typu „houba“ (SHA3).}


\clearpage
\section{Kvantový přenos informace - důvody použití, příklady protokolů.}


\clearpage
\section{Postkvantová kryptografie – důvody použití, jaké těžké matematické problémy se zde využívají, (kryptosystém McEliece, kryptosystém založený na mřížkách). Jednorázový podpis pomocí hašovacích funkcí (Lamport).}


\clearpage
\section{V souvislosti s nařízením eIDAS vysvětlete pojmy - elektronický podpis, zaručený elektronický podpis a kvalifikovaný elektronický podpis. elektronická pečeť, elektronické časové razítko.}


\clearpage
\section{Technologie blockchain – struktura, princip, možnosti využití.}


\clearpage
\section{Fyzicky neklonovatelné funkce (FNF) - k čemu lze využít, výhody a nevýhody, požadované vlastnosti, příklady.}


\clearpage
\section{Autentizační protokoly – na jakém principu pracují, využívané proměnné parametry, hodnocení jejich bezpečnosti (BAN logika).}
